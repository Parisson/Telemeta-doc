Since 2007, Parisson, a small french company,  has been developing a scalable and collaborative web platform for the management, access and automatic analysis of digital sound archives in the context of scientific research activities.

This web platform is based on Telemeta, an open-source web audio framework.
It carries out digital sound archives secure storing, indexing, geolocating  and publishing and simplifies the process of uploading and synchronizing large audio data.

Telemeta also transparently handles most multimedia file formats. For this purpose, it provides decoding, encoding and streaming features together with a smart embeddable HTML audio player that can display waveform or time-frequency representations of the signal.

Furthermore, the Telemeta platform focuses on the enhanced and collaborative user-experience in accessing audio items and their associated metadata.
It also provides to the expert users the possibility to further edit and enrich those metadata through global or time-based annotations.

One of the key features of the Telemeta architecture is to integrate TimeSide,  an external open-source audio processing framework written in Python.
TimeSide provides Telemeta with automatic signal processing and computational analysis capabilities. It also  wraps several audio features extraction libraries to provide a basis for automatic annotation, segmentation and acoustical analysis.

The TimeSide framework is designed to be flexible in terms of multimedia content analysis and researchers can easily plug their own audio analysis algorithms.
 
By putting together a digital sound archive management framework, a collaborative annotation framework and an audio signal analysis framework, Telemeta can constitute a very valuable and stimulating platform for bioacoustical, ecological and soundscape research.
Besides ensuring the preservation of the collected data, by publishing their sound archives, researchers can promote their works and make them accessible securely. They could also benefit from the interaction with peers and practitioners from different spheres and scientific disciplines.