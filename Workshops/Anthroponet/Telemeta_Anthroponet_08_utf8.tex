\documentclass[mathserif]{beamer}

% FOR BEAMER SLIDES
%\usepackage{palatino}
%\usepackage{helvet}
%\usepackage{eulervm}
\usepackage{mathpazo}
%\documentclass[sans,mathserif]{beamer}
%\usepackage{kerkis}
%\usepackage{kmath}
\usepackage{beamerthemeCambridgeUS}
%\usepackage{beamerthemeBoadilla}
%\usepackage{beamerthemeDarmstadt}
%\usepackage{beamerthemeGoettingen}
\usepackage{pgf,pgfarrows,pgfnodes,pgfautomata,pgfheaps,pgfshade}
\usepackage{colortbl}
\usepackage{amsmath,amssymb,amsfonts}
\usepackage{latexsym}
\usepackage{amsthm}
\usepackage{newlfont}
\usepackage{url}
\urlstyle{footnotesize}
\usepackage[french]{babel}
\usepackage[latin1]{inputenc}
\columnsep 1cm
\columnsep 1cm
\definecolor{rouge}{rgb}{0.7,0.1,0.1}
\newcommand{\chref}[2]{
    \href{#1}{\color{rouge}\underline{#2}}
}
\newcommand{\dchref}[1]{
    \href{#1}{\color{rouge}\underline{#1}}
}
\newcommand{\curl}[1]{
    \color{rouge}\underline{\url{#1}}
}
\title{\textbf{Projet Telemeta}}
\author{Anthroponet, CREM, MNHN, LAM, Parisson}
\institute{Guillaume Pellerin (Parisson SARL) \\
            Joséphine Simonnot (CREM)}
\date{\today\ - v0.3.4}

\begin{document}
\frame{\titlepage}


\section[Plan]{}
\frame{\tableofcontents}

\section{Introduction}

\subsection{Le fond d'archives du CREM}
\frame{\frametitle{Le fond d'archives du CREM}
Un fond ethnomusicologique unique :
\begin{itemize}
 \item 4291 collections (inédit et édité)
 \item Actuellement 22101 fiches documentaires comportant 50 champs en moyenne (60000 environ à terme)
 \item 3250 heures de sons inédits soit 4 To environ + montages et copies diverses
 \item 3500 heures de sons édités soit 4,5 To environ
 \item 300 heures environ de video ($\approx$ 4 To)
\end{itemize}
\vspace{0.5cm}
Besoins en numérisation, en traitement documentaire (reste 65\% du fond à effectuer) et en \textbf{valorisation} ! \\
}


\frame{\frametitle{Modèle de données du CREM}
  \begin{center}
   \vspace{-0.2cm}
   \pgfimage[width=12cm]{img/data_model2}
  \end{center}
}


\subsection{Vers la numérisation}
\frame{\tableofcontents[currentsubsection]}

\frame{\frametitle{L'archivage classique sur supports physiques}
Supports typiques : cylindres de cires, bandes magnétiques analogiques, disques vinyles, DAT, CD, etc...

\begin{columns}[c]

\column{6cm}
\begin{itemize}
 \item Buts :
    \begin{itemize}
    \item Conformité à l'oeuvre originale
    \item Reproductibilité
    \item Accessibilité
    \end{itemize}
 \item Problèmes :
    \begin{itemize}
    \item Fragilité : rayures, fissures, changement de phase du matériau, syndrôme du vinaigre,...
    \item Nécessité d'appareils de lecture specifiques
    \item Pérennité
    \end{itemize}
\end{itemize}
\column{6cm}
\framebox{\pgfimage[width=5cm]{img/CD-Moisi2}}
\framebox{\pgfimage[width=5cm]{img/CD-Moisi}}
\end{columns}

}

\frame{\frametitle{L'archivage sur disque dur}
\begin{itemize}
 \item Avantages :
    \begin{itemize}
    \item Information magnétique
    \item Confinement et compacité
    \item Vitesse et capacité d'accès (lecture et écriture)
    \item $\Rightarrow$ Valorisation
    \end{itemize}
 \item Inconvénients :
    \begin{itemize}
    \item Maintenance (recopie)
    \end{itemize}
\end{itemize}
\begin{center}
\pgfimage[width=5cm]{img/HDMAXTOR}
\end{center}
}


\section{Objectifs du projet Telemeta}
\frame{\tableofcontents[currentsection]}

\frame{\frametitle{Objectifs du projet Telemeta}
\begin{itemize}
 \item Pérenniser les archives audionumériques (logiciels et formats)
 \item Valoriser le patrimoine culturel par la consultation légale
 \item Optimiser la transmission des méta-données (OAI-PMH)
 \item Augmenter les capacités de Recherche (web sémantique, interopérabilité, croisement de données)
 \item Définir un systématisme de sauvegarde et de publication des oeuvres audiovisuelles
\end{itemize}
\vspace{3mm}
\begin{center}
$\Rightarrow$ Demande de financement à TGE Adonis (AAP 2009)\\
$\Rightarrow$ Demande de financement à l'ANR (AAP 2009)
\end{center}
\begin{center}
\begin{columns}[c]
\column{2.5cm}
\begin{center}
\pgfimage[width=2cm]{img/logo_telemeta_sat}\end{center}
\column{2.5cm}
\begin{center}
\pgfimage[width=2.5cm]{img/Logo-CREM-La.jpg}
\end{center}\column{2.5cm}
\begin{center}
\pgfimage[width=2.5cm]{img/titreMMSH2}\end{center}
\column{2cm}
\begin{center}
\pgfimage[width=1.5cm]{img/LAMlogonew03}\end{center}
\column{2.5cm}
\begin{center}
\pgfimage[width=2.5cm]{img/parisson_logo_200}\end{center}
\end{columns}
\end{center}
}


\section{Technologies}
\frame{\tableofcontents[currentsection]}

\frame{\frametitle{Telemeta : un projet libre et ouvert}
\textbf{Les fondamentaux du logiciel libre :}
\begin{itemize}
  \item Pérenniser les ressources informatiques
  \item Dynamiser le développement (partage, communautés internationales)
  \item Limiter les coûts de déploiement à grande échelle
\end{itemize}
\vspace{3mm}
\textbf{Briques 100\% 0pen Source :}
\begin{itemize}
 \item \chref{http://python.org}{Python} : langage \\
 \item \chref{http://djangoproject.com}{Django} : framework \\
 \item \chref{http://mysql.com}{MySQL} : base de données relationnelle \\
 \item \chref{http://linux.com}{Linux} : noyau serveur \\
 \item \chref{http://scipy.org}{Scipy}, \chref{http://www.ar.media.kyoto-u.ac.jp/members/david/softwares/audiolab/}{Audiolab} : traitement des signaux audio
 \item \chref{http://www.cecill.info}{CeCILL} : licence libre conforme au droit français
\end{itemize}
}

\frame{\frametitle{Les standards et normes utilisés par Telemeta}
\begin{itemize}
 \item Web :
    \begin{itemize}
     \item \chref{http://www.w3schools.com/html/}{HTML} : langage hypertextuel
     \item \chref{http://www.w3.org/Style/CSS/}{CSS} : styles
     \item \chref{http://fr.wikipedia.org/wiki/Structured_Query_Language}{SQL} : base de données
    \end{itemize}
 \item Audio :
    \begin{itemize}
    \item \chref{http://fr.wikipedia.org/wiki/WAVEform_audio_format}{WAV} : archivage audio brut
    \item \chref{http://fr.wikipedia.org/wiki/MP3}{MP3, MP4} : compression avec pertes, largement utilisé, encapsulation partielle
    \item \chref{http://www.vorbis.com}{OGG Vorbis} : compression avec pertes, open source, encapsulation totale
    \item \chref{http://fr.wikipedia.org/wiki/FLAC}{FLAC} : compression sans pertes, multi-pistes, open source, encapsulation totale
    \end{itemize}
 \item Métadonnées :
    \begin{itemize}
     \item \chref{http://en.wikipedia.org/wiki/XML}{XML} (W3C)
     \item \chref{http://dublincore.org/}{DublinCore} (OAI-PMH)
     \item \chref{http://mysql.com}{SQL}
     \item \chref{http://www.w3.org/TR/owl-features/}{OWL} : Web Ontology Language
    \end{itemize}
\end{itemize}
}

\frame{\frametitle{Plateforme communautaire de développement}
\begin{center}
\vspace{-0.2cm}
\dchref{http://telemeta.org}
   \framebox{\pgfimage[width=10cm]{img/telemeta_trac01}}
\end{center}
}

% \subsection{Architecture}
% \frame{\frametitle{L'architecture de Telemeta}
%   \begin{center}
%    \vspace{-0.2cm}
%    \pgfimage[width=11cm]{img/architecture_fr}
%   \end{center}
% }


\section{Développement}
\frame{\tableofcontents[currentsection]}

\subsection{Version 0.3.2}

\frame{\frametitle{La base de données du CREM intégrée à Telemeta}
\begin{center}

\dchref{http://crem.parisson.com}
\end{center}
  \begin{center}
   \vspace{-0.3cm}
   \pgfimage[width=11cm]{img/crem_telemeta-0.3.2}
  \end{center}
}


% \subsection{Démonstration}
% \frame{\tableofcontents[currentsubsection]}


\subsection{Besoins}
\frame{\frametitle{Objectifs pour une version 1.0 de production}
Elements à programmer et/ou consolider pour la version 1.0 de production
\begin{itemize}
 \item Intégration du \textbf{\textit{workflow}} (gestion de publication)
 \item \textbf{Thésaurus} thématiques, \textbf{ontologies} de recherche
 \item \textbf{Sauvegarde} externe (audio + meta-données)
 \item Compléter la vue \textbf{DublinCore} (moissonnage par méta-portails)
 \item \textbf{Marqueurs} temporels, lecture dynamique avec pièces attachées (texte, images)
 \item \textbf{Analyse des signaux audio} (niveaux, transitoires, voisins fréquentiels,...)
 \item \textbf{Specifications} et \textbf{documentation} pour les composants (API)
 \item Fonctions de test
\end{itemize}
}


\frame{\frametitle{En route pour 0.4 (edition)}
  \begin{center}
   \vspace{-0.2cm}
   \pgfimage[width=9.5cm]{img/maquette_v0.4_item_edit}
  \end{center}
}


\frame{\frametitle{En route pour 0.4 (vue item)}
  \begin{center}
   \vspace{-0.2cm}
   \pgfimage[width=9.5cm]{img/maquette_v0.4_item_view}
  \end{center}
}



\frame{\frametitle{Partenariats}
Partenaires participants :
\begin{itemize}
 \item Centre de Recherche en Ethnomusicologie (\textbf{CREM}) du Laboratoire d'Ethnologie et de Sociologie Comparée (\textbf{LESC}), UMR 7186
 \item Equipe Lutheries, Acoustique et Musique (\textbf{LAM}) de l'Institut Jean le Rond d'Alembert (\textbf{IJLRA}), UMR 7190
 \item Médiathèque Eric-de-Dampierre de la \textbf{MAE}, Nanterre
 \item Museum National d'Histoire Naturelle (\textbf{MNHN})
 \item Musée des Civilisations de l'Europe de la Méditerranée (\textbf{MuCEM})
 \item Phonothèque de la Maison Méditerranéenne des Sciences de l'Homme (\textbf{MMSH})
\end{itemize}
Partenaires potentiels :
\begin{itemize}
  \item Bibliothèque Nationale de France (\textbf{BNF})
  \item Institut National de l'Audiovisuel (\textbf{INA})
  \item Institut de Recherche et d'Innovation (\textbf{IRI})
  \item Queen Mary University (\textbf{QMU}, Londres)
\end{itemize}
}


\section{Conclusion et perspectives}
\frame{\frametitle{Conclusion et perspectives}
\begin{itemize}
 \item Technologie prometteuse pour la \textbf{sauvegarde et la valorisation} du patrimoine audio
 \item \textbf{Déploiement} et \textbf{pérennité} optimisés grâce à l'\textbf{Open Source}
 \item \textbf{Intégration souple} de données ``métiers'' hétérogènes (sciences humaines et sciences informatiques)
 \item Plateforme de \textbf{développement} ouverte
 \item Système de \textbf{composants} facilitant l'intégration de nouvelles fonctions et l'export des fonctions vers d'autres logiciels.
 \item Nécessite un \textbf{financement} pour le développement et le déploiement des ressources à un niveau national et international
\end{itemize}
}

\frame{\frametitle{Merci !}
 \begin{center}
   \vspace{-0.2cm}
    \pgfimage[width=3cm]{img/logo_telemeta_sat}\\
    \dchref{http://telemeta.org}\\
    \vspace{10mm}
    \pgfimage[width=3.5cm]{img/parisson_logo_200}\\
    \dchref{http://parisson.com}\\
    \vspace{10mm}
    \pgfimage[width=1cm]{img/88x31.png}\\
    Ce document est mise à disposition sous un \chref{http://creativecommons.org/licenses/by-nc-sa/2.0/fr/}{contrat Creative Commons}.
  \end{center}
}


\end{document}