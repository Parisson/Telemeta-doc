\documentclass[mathserif]{beamer}

% FOR BEAMER SLIDES
%\usepackage{palatino}
%\usepackage{helvet}
%\usepackage{eulervm}
\usepackage{mathpazo}
%\documentclass[sans,mathserif]{beamer}
%\usepackage{kerkis}
%\usepackage{kmath}
\usepackage{beamerthemeCambridgeUS}
%\usepackage{beamerthemeBoadilla}
%\usepackage{beamerthemeDarmstadt}
%\usepackage{beamerthemeGoettingen}
\usepackage[utf8]{inputenc}
\usepackage[]{fontenc}
\usepackage[francais]{babel}
\usepackage{pgf,pgfarrows,pgfnodes,pgfautomata,pgfheaps,pgfshade}
\usepackage{colortbl}
\usepackage{amsmath,amssymb,amsfonts}
\usepackage{latexsym}
\usepackage{amsthm}
\usepackage{newlfont}
\usepackage{url}
\urlstyle{footnotesize}
\columnsep 1cm
\definecolor{rouge}{rgb}{0.7,0.1,0.1}
\newcommand{\chref}[2]{
    \href{#1}{\color{rouge}\underline{#2}}
}
\newcommand{\dchref}[1]{
    \href{#1}{\color{rouge}\underline{#1}}
}
\newcommand{\curl}[1]{
    \color{rouge}\underline{\url{#1}}
}
\setbeamertemplate{blocks}[center=true]

\usepackage{texments}

\title{\textbf{Projet Telemeta}}
\institute{Guillaume Pellerin (Parisson)}
\date{23/04/2012 - CMTRA - RIZE}

\begin{document}
\frame{\titlepage
\vspace{-1cm}
\begin{center}
\pgfimage[width=3cm]{img/logo_telemeta_800}
 \end{center}
}

\section[Plan]{}
\frame{\tableofcontents}

\section{Introduction}

\subsection{Présentations}
\frame{\tableofcontents[currentsubsection]}

\subsubsection*{Parisson}
\frame{\frametitle{Parisson : solutions média sur-mesure}
Les domaines de recherche et développement :
\begin{itemize}
	\item \textbf{Captation et enregistrement} : captation multi-canal audio et vidéo, enregistrements et
	diffusions automatisés (projet TeleCaster)
	\item \textbf{Traitement des signaux} : analyse multi-dimensionnelle de signaux musicaux et vocaux,
	indexation semi-automatique (projets TimeSide, GStreamer)
	\item \textbf{Architectures, réseaux et base de données} : gestion des archives numériques et des
	métadonnées structurées, flux de métadonnées, streaming en temps-réel (Projets Telemeta,
	DeeFuzzer)
	\item \textbf{Interfaces} : définition et production des outils de navigation dans les flux de données
	(Technologies AJAX, JSON, CSS3, HTML5)\\
\end{itemize}
}
\frame{\frametitle{Parisson : solutions média sur-mesure}
\textbf{Les domaines d'application :}
\begin{itemize}
	\item chaînes de production multimédia
	\item télé-conférence, e-leaning
	\item archivage pérenne sémantique
	\item indexation collaborative
	\item gestion et hébergement de fonds (laboratoires, web radios, services audio)
\end{itemize}
\vspace{0.3cm}
\textbf{Clients :} Laboratoires, musées, mediathèques, centres de formation, industrie musicale\\
\vspace{0.3cm}
\textbf{Mots clés :} audio, vidéo, métadonnées, acoustique, enregistrement, transcodage, diffusion, indexation, analyse audio, archivage pérenne, broadcast, streaming, e-learning, réseaux sociaux.\\
\vspace{0.3cm}
\textbf{Contexte de développement} : 100$\%$ open source, logiciels libres, Linux, Python, Debian.
}


\subsection{Objectifs du projet Telemeta}
\frame{\tableofcontents[currentsubsection]}

\frame{\frametitle{Telemeta : open web audio CMS}
Objectifs du projet :
\begin{itemize}
 \item \textbf{Pérenniser} les archives audionumériques (logiciels et formats)
 \item \textbf{Valoriser} le patrimoine culturel par la consultation légale
 \item Faciliter et \textbf{optimiser} la transmission des méta-données
 \item Augmenter les capacités de recherche et de travail collaboratif (saisie en ligne, \textbf{géo-localisation}, \textbf{web sémantique}, \textbf{interopérabilité}, croisement de données)
 \item Définir une ergonomie d'indexation et de \textbf{publication collaborative}, un principe de sauvegarde des oeuvres musicales
\end{itemize}
\vspace{3mm}
}

\subsection{Historique}
\frame{\tableofcontents[currentsubsection]}

\frame{\frametitle{Historique}
\begin{itemize}
\item 2007 : rencontre entre le Centre de Recherche en Ethnomusicologie, Parisson et le LAM, contour des besoins mutuels, soutiens du projet Anthroponet
\item 2008 : définition du cahier des charges, choix du framework, modélisation, migrateur 4D $>$ MySQL
\item 2009 : prototypage (v0.5), recherche de partenaires
\item 2010 : corrections, développement agile (v0.9), soutien du TGE Adonis
\item 2011 : migration définitive, v1.0 déployée sur serveur, mise en production au CREM
\item 2012 : développement de fonctions utilisateurs, avancées, médias associés, passage à la vidéo \\
\end{itemize}

\begin{center}
\begin{columns}[c]
\column{2.5cm}
\begin{center}
\pgfimage[width=2cm]{img/tge_logo}\end{center}
\column{2.5cm}
\begin{center}
\pgfimage[width=2.5cm]{img/Logo-CREM-La.jpg}
\end{center}
\column{2.5cm}
\begin{center}
\pgfimage[width=2.5cm]{img/parisson_logo_200}\end{center}
\column{2cm}
\begin{center}
\pgfimage[width=1.2cm]{img/logo-mnhn}\end{center}
\end{columns}
\end{center}
}


\frame{\frametitle{Le Centre de Recherche en Ethnomusicologie}
Un fond ethnomusicologique unique :
\begin{itemize}
 \item 4291 collections (inédit et édité)
 \item Actuellement 22101 fiches documentaires comportant 50 champs en moyenne (60000 environ à terme)
 \item 3250 heures de sons inédits soit 4 To environ + montages et copies diverses
 \item 3500 heures de sons édités soit 4,5 To environ
 \item 300 heures environ de video ($\approx$ 4 To)
\end{itemize}
\vspace{0.5cm}
Depuis Telemeta 1.0 : plus de 52000 révisions, 1 To environ importés.\\
\begin{center}
\dchref{http://archives.crem-cnrs.fr}
\end{center}
}


\frame{\frametitle{Partenariats}
Partenaires participants :
\begin{itemize}
 \item Centre de Recherche en Ethnomusicologie (\textbf{CREM}) du Laboratoire d'Ethnologie et de Sociologie Comparée (\textbf{LESC}), UMR 7186
 \item Equipe Lutheries, Acoustique et Musique (\textbf{LAM}) de l'Institut Jean le Rond d'Alembert (\textbf{IJLRA}), UMR 7190
 \item Médiathèque Eric-de-Dampierre de la \textbf{MAE}, Nanterre
 \item Museum National d'Histoire Naturelle (\textbf{MNHN})
 \item Institut de Recherche et d'Innovation (\textbf{IRI})
 \item Musée des Civilisations de l'Europe de la Méditerranée (\textbf{MuCEM})
 \item Phonothèque de la Maison Méditerranéenne des Sciences de l'Homme (\textbf{MMSH})
\end{itemize}
}

\section{Présentation du logiciel}

\subsection{Technologies et communautés}
\frame{\tableofcontents[currentsubsection]}

\frame{\frametitle{Telemeta : un projet libre et ouvert}
\textbf{Les fondamentaux du logiciel libre :}
\begin{itemize}
  \item Pérenniser les ressources informatiques
  \item Dynamiser le développement (partage, communautés internationales)
  \item Limiter les coûts de déploiement à grande échelle
  \item Se conformer aux standards ouverts d'Internet
\end{itemize}
\vspace{3mm}
\textbf{Briques libres de Telemeta :}
\begin{itemize}
 \item \chref{http://python.org}{Python} et \chref{http://djangoproject.com}{Django} : langage côté serveur et framework web \\
 \item \chref{http://code.google.com/p/timeside}{TimeSide} : traitement des signaux audio, lecteur audio HTML5 dynamique
 \item \chref{http://mysql.com}{MySQL} : base de données relationnelle \\
 \item \chref{http://linux.com}{Linux} : noyau serveur \\
 \item \chref{http://www.cecill.info}{CeCILL} : licence libre conforme au droit français (CNRS, INRIA, CEA)
\end{itemize}
}


\frame{\frametitle{L'architecture de Telemeta}
  \begin{center}
   \vspace{-0.2cm}
   \pgfimage[width=11cm]{img/architecture_fr}
  \end{center}
}

\subsection{Modèle de données}
\frame{\tableofcontents[currentsubsection]}

\frame{\frametitle{Modèle de données du CREM}
  \begin{center}
   \vspace{-0.2cm}
   \pgfimage[width=12cm]{img/data_model2}
  \end{center}
}

\subsection{Interface web}
\frame{\tableofcontents[currentsubsection]}

\frame{\frametitle{Démonstration !}
\dchref{http://demo.telemeta.org}
}

\subsection{Droits d'accès}
\frame{\tableofcontents[currentsubsection]}

\frame{\frametitle{Droits d'accès}
\begin{itemize}
 \item \textbf{Utilisateurs} : profils, droits, gestion des mots de passe
 \item \textbf{Groupes} : administrateur, documentaliste, chercheur, membre, anonyme
 \item \textbf{Droits} : ajouter, supprimer ou modifier les objets selon les règles
 \item Pour tous les utilisateurs : \textbf{listes de lecture} personnelle, ajout de \textbf{marqueurs temporels}
 \item \textbf{Droits particuliers} pour la lecture audio (date glissante de 50 ans, paramètre ``public access'')
\end{itemize}
}

\frame{\frametitle{Détail des droits des groupes}
\begin{itemize}
 \item \textbf{Anonyme} : parcours du site, lecture selon les autorisations de chaque objet
 \item \textbf{Membre} : ajout de listes de lectures personnelles, ajout de marqueurs
 \item \textbf{Chercheur} : ajout et edition des fiches documentaires
 \item \textbf{Documentaliste} : suppression d'objets, tous droits sur la base de données (hors utilisateurs), lecture audio de tous les items
 \item \textbf{Administrateur} : ajouter des utilisateurs
\end{itemize}
\vspace{5mm}
+ droits spéciaux au cas par cas :
\begin{itemize}
 \item Lecture audio de tous les items
 \item Téléchargement de tous les items
\end{itemize}

}

\subsection{Traitement et lecture audio}
\frame{\tableofcontents[currentsubsection]}

\frame{\frametitle{TimeSide : Web Audio Components}
  \begin{center}
   \vspace{-0.2cm}
   \pgfimage[width=9.5cm]{img/timeside_schema}
  \end{center}
}

\frame{\frametitle{TimeSide UI : lecteur audio HTML dynamique}
\begin{itemize}
\item Lecture audio
\item Analyse du signal
\item Indexation temporelle
\item Portabilité
\item \chref{http://code.google.com/p/timeside/wiki/UiGuide}{code.google.com/p/timeside/wiki/UiGuide}
\end{itemize}

\begin{center}
\pgfimage[width=5cm]{img/player_01}
\end{center}

}

\subsection{Moissonnage et sauvegarde}
\frame{\tableofcontents[currentsubsection]}

\frame{\frametitle{Moissonnage et sauvegarde}
\begin{itemize}
\item mapping Dublin Core
\item serveur OAI-PMH intégré
\item flux RSS (revisions)
\item sauvegarde sérialisée : XML + WAV
\item upload accumulatif
\end{itemize}
}

\subsection{Roadmap}
\frame{\tableofcontents[currentsubsection]}

\frame{\frametitle{Roadmap}
Objectifs pour Telemeta 1.x:
\begin{itemize}
 \item \textbf{Compatibilté HTML5} totale du lecteur audio
 \item \textbf{Accès spéciaux} par adresses IP
 \item Intégration et MAJ des \textbf{ontologies} (langues, géographie)
\end{itemize}
\vspace{0.3cm}
Objectifs pour Telemeta 2.x:
\begin{itemize}
 \item \textbf{Modèle générique de ressources} de données
 \item \textbf{Fonctions d'analyse} augmentée (reconnaissance, recoupement statistique) : ANR CONTINT 2011 \textbf{DIADEMS} et ANR CORPUS 2011 \textbf{DicA3Ref}
 \item \textbf{Déploiement mutualisé} avec l'IN2P3 (CNRS)
 \item Définition d'une \textbf{API} cliente accessible au format JSON pour toutes les méta-données
\end{itemize}
}


\section{Conclusion et perspectives}
\frame{\tableofcontents[currentsubsection]}

\frame{\frametitle{Conclusion et perspectives}
\begin{itemize}
 \item Technologie et ergonomie prometteuse pour la \textbf{sauvegarde et la valorisation} du patrimoine musical
 \item \textbf{Déploiement} et \textbf{pérennité} optimisés avec les briques open source
 \item \textbf{Intégration souple} de données métiers hétérogènes (sciences humaines et sciences informatiques)
 \item \textbf{Plateforme collaborative} à un niveau international
 \item Plateforme sociale d'indexation : (au CREM, par ex, plus de \textbf{450} fiches en moyenne éditées par semaines !)
 \item Plateforme de \textbf{développement} ouverte
\end{itemize}
}



\frame{\frametitle{Merci !}
 \begin{center}
   \vspace{-0.2cm}
    \pgfimage[width=4cm]{img/logo_telemeta_800}\\
    \dchref{http://telemeta.org}\\
   \vspace{1cm}
  \pgfimage[width=4cm]{img/parisson_logo_200}\\
  guillaume.pellerin@parisson.com\\
  @parisson\_studio\\
    \vspace{10mm}
    \tiny{Ce document est mise à disposition selon un \chref{http://creativecommons.org/licenses/by-nc-sa/2.0/fr/}{contrat Creative Commons}}
  \end{center}
}

\frame{\frametitle{Les standards et normes utilisés par Telemeta}
\begin{itemize}
 \item Web :
    \begin{itemize}
     \item \chref{http://www.w3schools.com/html/}{HTML5} : langage hypertextuel avec balises $<$audio$>$ $<$video$>$
     \item \chref{http://www.w3.org/Style/CSS/}{CSS} : styles
     \item \chref{http://fr.wikipedia.org/wiki/JavaScript}{JavaScript} : langage côté navigateur (interfaces et lecteur dynamiques)
    \end{itemize}
 \item Audio :
    \begin{itemize}
    \item \chref{http://fr.wikipedia.org/wiki/WAVEform_audio_format}{WAV} : archivage audio brut
    \item \chref{http://fr.wikipedia.org/wiki/MP3}{MP3, MP4} : compression avec pertes, largement utilisé, encapsulation partielle
    \item \chref{http://www.vorbis.com}{OGG Vorbis} : compression avec pertes, open source, encapsulation totale
    \item \chref{http://fr.wikipedia.org/wiki/FLAC}{FLAC} : compression sans pertes, multi-pistes, open source, encapsulation totale
    \end{itemize}
 \item Métadonnées :
    \begin{itemize}
     \item \chref{http://en.wikipedia.org/wiki/XML}{XML} (W3C)
     \item \chref{http://dublincore.org/}{DublinCore} (OAI-PMH)
     \item \chref{http://fr.wikipedia.org/wiki/Structured_Query_Language}{SQL} : base de données
 %    \item \chref{http://www.w3.org/TR/owl-features/}{OWL} : Web Ontology Language
    \end{itemize}
\end{itemize}
}



\end{document}