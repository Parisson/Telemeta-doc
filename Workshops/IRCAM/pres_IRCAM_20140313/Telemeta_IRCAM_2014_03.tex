\documentclass[10pt, final, hyperref, table]{beamer}
\mode<presentation>


 %\usepackage[english]{babel} % "babel.sty"
% \usepackage{french}                  % "french.sty"
%  \usepackage{franglais}               % "franglais.sty" (a defaut)
  \usepackage{times}            % ajout times le 30 mai 2003
 
%% --------------------------------------------------------------
%% CODAGE DE POLICES ?
%% Si votre moteur Latex est francise, il est conseille
%% d'utiliser le codage de police T1 pour faciliter la césure,
%% si vous disposez de ces polices (DC/EC)
\usepackage[utf8]{inputenc}
\usepackage[T1]{fontenc}
\usepackage{eurosym}


%% ==============================================================
%\usepackage{graphicx}
\usepackage{amsmath,amsfonts}
%\usepackage[table]{xcolor}
\usepackage{subfigure}
\usepackage{fancybox}

\usepackage{multicol}
\usepackage{wrapfig}
\usepackage{listings}
\usepackage{xcolor}
\usepackage{multimedia} % For playing sound

\usepackage{hyperref}
% Define hyperlinks color
\definecolor{links}{HTML}{2A1B81}
\hypersetup{colorlinks,linkcolor=,urlcolor=links}

\usetheme{Madrid}
\setbeamercovered{transparent}


% telemeta red
\definecolor{telemetaRed}{rgb}{0.41568, 0.01176, 0.02745}   % #6A0307
\usecolortheme[rgb={0.41568, 0.01176, 0.02745}]{structure} 

%\setbeamercolor{frametitle}{bg=telemetaRed}
% Display a grid to help align images
%\beamertemplategridbackground[1cm]

%We will get the normal bibliography style (number or text instead of icon) by including the following code
\setbeamertemplate{bibliography item}[text]
\setbeamerfont{caption}{size=\footnotesize}
% listings settings
\definecolor{lstComments}{rgb}{0,0.6,0}
\definecolor{lstBkgrd}{rgb}{0.95,0.95,1}
\lstset{%
  language=Python, % the language of the code
  frame=single,  % adds a frame around the code
  frameround=tttt,
  commentstyle=\color{lstComments},% comment style
  backgroundcolor=\color{lstBkgrd},   % choose the background color
  basicstyle=\tiny,       % the size of the fonts that are used for the code
  stringstyle=\ttfamily,  % typewriter type for strings
  keywordstyle=\color{blue},      % keyword style
  showstringspaces=false,          % underline spaces within strings only
}


\definecolor{rouge}{rgb}{0.7,0.1,0.1}
\newcommand{\chref}[2]{
    \href{#1}{\color{rouge}\underline{#2}}
}
\newcommand{\dchref}[1]{
    \href{#1}{\color{rouge}\underline{#1}}
}
\newcommand{\curl}[1]{
    \color{rouge}\underline{\url{#1}}
}
\title[TimeSide]{Telemeta\\ open web audio content management system}

\author{Guillaume Pellerin\inst{1}, Thomas Fillon \inst{1,2}}


\institute[Parisson]{
  \inst{1}%
  Parisson, Paris, France\\
  \inst{2}%
  LAM, Institut Jean Le Rond d'Alembert, UPMC Univ. Paris 06, UMR CNRS 7190, Paris, France\\
\vskip1ex
 \begin{center}
   \includegraphics[width=.3\linewidth]{img/parisson_logo_FINALE_com.pdf}
 \end{center}
}
\date{IRCAM - WAVE \\ 13/03/2014}        

\begin{document}
\frame{\titlepage}


\section[Table of contents]{}
\frame{\frametitle{Table of contents}
\tableofcontents
}

\section{Goals}
\frame{\frametitle{Main goals}
\begin{itemize}
  \item Save, scale and sustain big music data and related metadata
  \item Play audio and read metadata at the same time, synchronously
  \item Index and share music data through a collaborative web app
  \item Link music data to various ontologies and external services
  \item Manage access rules and copyrights easily through time
  \item Process audio on demand through a modular architecture
\end{itemize}
}

\section{History of the project}
\frame{\frametitle{History of the project}
\begin{itemize}
  \item 2006 : definition of the goals (open source web audio collaborative platform)
  \item 2007 : first partner : CREM
  \item 2007 - 2009 : technical specifications, definition of the DB migrator
  \item 2008 : prototype development
  \item 2008 - 2010 : workflow and format specifications 
  \item 2011 : development, final migration and release of Telemeta 1.0 to the CREM for production : \dchref{http://archives.crem-cnrs.fr}
  \item 2011 - 2014 : collaborative indexing, more development, massive data imports...
\end{itemize}
}

\section{Technologies}
\frame{\frametitle{Technologies}
\textbf{100\% 0pen Source}
\begin{itemize}
 \item \chref{http://python.org}{Python} : smart object oriented language \\
 \item \chref{http://djangoproject.com}{Django} : high-level web MVC framework \\
 \item \chref{https://github.com/yomguy/TimeSide}{TimeSide} : open web audio processing framework 
 \item \chref{http://gstreamer.freedesktop.org/}{GStreamer} : open source multimedia framework
 \item MySQL, PostgreSQL, etc : relational databases \\
 \item GNU / Linux : application and library suite / kernel \\
\end{itemize}
}

\section{Key features}
%\frame{\tableofcontents[currentsection]}
\frame{\frametitle{Features}
      \begin{itemize}
      \item \alert{Pure HTML5} web user interface including dynamical forms
        and smart workflows
      \item \alert{On the fly} audio analyzing, transcoding and \alert{metadata}
        embedding in various formats
      \item Social editing with \alert{semantic ontologies}, smart workflows,
        realtime tools, human or automatic \alert{annotations and
        segmentations}
      \item User management with individual desk, playlists, profiles
        and access rights
      \item High level geo-located search engine
      \item DublinCore, OAI-PMH, RSS, XML and JSON \alert{data providers}
      \item Multi-language support (now english and french)
      \end{itemize}
}


\section{Data model}
%\frame{\tableofcontents[currentsection]}
\frame{\frametitle{Data model}
\begin{itemize}
 \item Main objects (hierarchy)
    \begin{center}
    \pgfimage[width=10cm]{img/TM_model}
    \end{center}
 \item Complete model : \chref{http://telemeta.org/export/cb1fc9ad29e2cb0122fde9dd76c5275ba5dc7d14/doc/devel/telemeta-all.pdf}{view PDF}
\end{itemize}

}


% \subsection{Démonstration}
% \frame{\tableofcontents[currentsubsection]}

\section{Related projects}
\frame{\frametitle{Related projects}
\begin{itemize}
\item \chref{https://github.com/yomguy/TimeSide}{TimeSide} : open web audio processing framework
      \begin{itemize}
      \item Easy plugin architecture (full \alert{Python})
      \item Asynchronous and fast audio processing
      \item \alert{Feature extraction} (Aubio, Yaafe, Vamp plugins)
      \item Smart, fancy and dynamical \alert{HTML5 web audio player}
      \end{itemize}
\vspace{0.5cm}
\item \chref{http://www.irit.fr/recherches/SAMOVA/DIADEMS/fr/welcome/&cultureKey=en}{DIADEMS} : Description, Indexation, Access to Sound and Ethnomusicological Documents
    \begin{itemize}
    \item granted by ANR : french national research agency (ANR-12-CORD-0022)     
    \item 3 years, 8 partners, 850 k\euro
    \item new collaboration between human and computer science laboratories (not so easy!)
    \item apply MIR algorithms on large scale ethnomusicological data
    \item define some high level interfaces to find musical informations in complex corpus
    \item \dchref{http://diadems.telemeta.org}
    \end{itemize}
\end{itemize}
}

\section{Development}
%\frame{\tableofcontents[currentsection]}
\frame{\frametitle{Development}
\begin{itemize}
 \item Links
    \begin{itemize}
    \item \dchref{http://telemeta.org}
    \item \dchref{https://github.com/yomguy/Telemeta/}
    \item \dchref{https://github.com/yomguy/TimeSide/}
    \end{itemize}
 \item Team
    \begin{itemize}
     \item Guillaume Pellerin
     \item Thomas Fillon
     \item Paul Brossier
     \item Riccardo Zaccarelli
     \item Maxime Lecoz
     \item David Doukan
    \end{itemize}
 \item Licence : CeCILL v2 (GPL v2 compatible)
\end{itemize}
}


\frame{\frametitle{Partners}
\begin{itemize}
\item Sponsors:
\begin{itemize}
 \item CNRS
 \item Huma-Num (ex TGE Adonis)
 \item ANR
 \item CREM
 \item UPMC
 \item Parisson
\end{itemize}

\vspace{0.25cm}

\item Partners :
\begin{itemize}
\item IRIT (université Paul Sabatier, Toulouse 3)
\item LIMSI (universités Pierre et Marie Curie (UPMC, Paris 6) et Paris-Sud)
\item LAM (institut Jean Le Rond d'Alembert, UPMC)
\item LABRI (université de Bordeaux)
\item CREM (université Paris Ouest Nanterre La Défense)
\item LESC (université Paris Ouest Nanterre La Défense)
\item Museum d'Histoire Naturelle de Paris
\item Musée du Quai Branly
\end{itemize}

\end{itemize}

\begin{center}
\begin{columns}[c]
\column{2.5cm}
\begin{center}
\pgfimage[width=1cm]{img/logo-CNRS}\end{center}
\column{2.5cm}
\begin{center}
\pgfimage[width=2.5cm]{img/Logo-CREM-La.jpg}
\end{center}
\column{2.5cm}
\begin{center}
\pgfimage[width=2.5cm]{img/parisson_logo_200}\end{center}
\column{2cm}
\begin{center}
\pgfimage[width=1.2cm]{img/logo-mnhn}\end{center}
\end{columns}
\end{center}

}


\section{DIADEMS}
\frame{\frametitle{DIADEMS : roadmap}
\begin{center}
\pgfimage[width=12cm]{img/TM_Roadmap}
\end{center}
}

\frame{\frametitle{DIADEMS : TODO list}
\begin{itemize}
 \item TimeSide
    \begin{itemize}
    \item web server (django)
    \item process task manager
    \item full HTML5 zooming player (+ annotations, segmentations, etc..)
    \item analyzer parameters (+ interface)
    \item more filtering (FIR, IIR, phase vocoder)
    \item \dchref{https://github.com/yomguy/TimeSide/issues}
    \end{itemize}
 
 \vspace{0.5cm}
 
 \item Telemeta
    \begin{itemize}
    \item class based views
    \item rewrite geolocation services
    \item public and enhanced user playlists
    \item smart breadcrumbs 
    \item better interactions with TimeSide
    \item \dchref{http://telemeta.org/report/1}
    \end{itemize}

\end{itemize}

}



\frame{\frametitle{The end}
 \begin{center}
   \large{Thanks !} \\
   \vspace{0.5cm}
    \chref{http://telemeta.org}{telemeta.org}
  \end{center}
}

\end{document}