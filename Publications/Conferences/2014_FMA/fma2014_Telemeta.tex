\documentclass{paper}
%\hyphenation{Post-Script}
\usepackage[authoryear]{natbib}
%\bibliographystyle{plainnat}
\usepackage{fma2014}

\usepackage{graphicx}
%\usepackage{amssymb}

%\usepackage{xcolor}
%\usepackage{hyperref} % Apparemment pas compatible avec le style AES !!
\usepackage{url}
\usepackage[utf8]{inputenc}
\usepackage[T1]{fontenc}
%\usepackage{enumitem}
%\setlist{nosep}

%\setlength{\parskip}{0pt}

%----------
%  Header
%----------


% first the title is needed
\title{An open web audio platform for ethnomusicological sound archives management and automatic analysis}

% a short form should be given in case it is too long for the running head
%\shorttitle{Telemeta : open web audio platform for sound archives}

\threeauthors
  {Thomas Fillon} {
    PARISSON / LAM, \\
Institut Jean Le Rond d'Alembert, \\
     UPMC Univ. Paris 06, \\
  UMR CNRS 7190\\
   %PARISSON\\
 {\tt  thomas.fillon@parisson.com}}
 %\href{http://www.parisson.com}{PARISSON}, 16 rue Jacques Louvel-Tessier 75010 Paris, France
 %\url{{thomas.fillon,guillaume.pellerin}@parisson.com}}
  {Guillaume Pellerin, Paul Brossier} {
    PARISSON \\
16 rue Jacques Louvel-Tessier \\
 Paris, France\\ \vspace{0.3cm}
    {\tt guillaume.pellerin@parisson.com}}
  {Jos{\'e}phine Simonnot} {CREM, LESC, UMR CNRS 7186\\MAE, Université Paris Ouest\\ Nanterre La Défense
{\tt %josephine.simonnot@mae.u-paris10.fr
}\thanks{This work is partly supported by a grant from the french National Research Agency (ANR) with reference ANR-12-CORD-0022.}\vspace{0.7cm}}



\begin{document}
%
\maketitle
%
\begin{abstract}
Since 2007, ethnomusicologist and engineers have joint their effort to develop a collaborative web platform for management of and access to digital sound archives. This platform has been deployed since 2010 and hold the archives of the \emph{Center for Research in Ethnomusicology}, which is the most important collection in Europe.
This web platform is based on Telemeta, an open-source web audio framework dedicated to digital sound archives secure storing, indexing and publishing. It focuses on the enhanced and collaborative user-experience in accessing audio items and their associated metadata and on the possibility for the expert users to further enrich those metadata.

 Telemeta architecture relies on \emph{TimeSide}, an open audio processing framework written in Python which provides decoding,  encoding and streaming methods for various formats together with a smart embeddable HTML audio player. TimeSide also includes a set of audio analysis plugins and additionally wraps several audio features extraction libraries to provide automatic annotation, segmentation and musicological analysis.

\end{abstract}

\section{Introduction}\label{sec:intro}

 % In social sciences like anthropology and linguistics, researchers have to work on multiple types of multimedia documents such as photos, videos, sound recordings or databases. The need to easily access, visualize and annotate such materials can be problematic given their diverse formats, sources and given their chronological nature.
  %With this in mind, some laboratories\footnote{The Research Center on Ethnomusicology (CREM), the Musical Acoustics Laboratory (LAM, UMR 7190) and the sound archives of the Mediterranean House of Human Sciences (MMHS)} involved in ethnomusicological research have been working together on that issue.

  In the context of ethnomusicological research, the Research Center on Ethnomusicology (CREM) and Parisson, a company specialized in the management of audio databases, have been developing an innovative, collaborative and interdisciplinary open-source web-based multimedia platform since 2007. 
This platform, \emph{Telemeta} is designed to fit the professional requirements from both sound archivists and researchers in ethnomusicology. The first prototype of this platform has been online\footnote{Archives sonores du CNRS, Musée de l'Homme, http://archives.crem-cnrs.fr} since 2008 and is now fully operational and used on a daily basis for ethnomusicological studies.

The benefit of this platform for ethnomusicological research has been describe in several publications \citep{Simmonot_IASA_2011, Julien_IASA_2011, Simonnot_ICTM_2014}.

Recently, an open-source audio analysis framework, \emph{Timeside}, has been developed to bring automatic music analysis capabilities to the web platform and thus have turned Telemeta into a complete resource for \emph{Computational Ethnomusicology} \citep{Tzanetakis_2007_JIMS, Gomez_JNMR_2013}

 \section{The Telemeta platform}\label{sec:Telemeta}
 \subsection{Web audio content management features and architecture}
The primary purpose of Telemeta was to provided ethnomusicological community with a platform to access, preserve and distribute audio research materials together with their associated metadata, as these data provide key information on the context and significance of the recording.

 Telemeta\footnote{http://telemeta.org} is a free and open source\footnote{Telemeta code is available under the CeCILL Free Software License Agreement \texttt{http://cecill.info/licences/Licence\_CeCILL\_V2-en.html}} web audio platform which introduces efficient and secure methods for back-uping, indexing, transcoding, analyzing, publishing and visualizing any digitized audio or video file with its metadata.

The time-based nature of such audio-visual materials and some associated metadata as annotations raises issues of access and visualization. Easy access to these data, as you listen to the recording, represents a significant improvement.

 An overview of the Telemeta's web interface is illustrated in Figure~\ref{fig:Telemeta} and Telemeta architecture is represented in Figure~\ref{fig:TM_arch}
 \begin{figure}
   \centering
   \fbox{\includegraphics[width=0.97\linewidth]{img/telemeta_screenshot_en.png}}
   \caption[1]{Screenshot excerpt of the \emph{Telemeta} web interface}
    \label{fig:Telemeta}
 \end{figure}

\begin{figure*}[htbp]
  \centering
  \includegraphics[width=0.5\linewidth]{img/TM_arch.pdf}
  \caption{Telemeta architecture}\label{fig:TM_arch}
\end{figure*}

Telemeta has been designed for professionals who wants to easily organize, backup, archive and publish documented sound collections of audio files, CDs, digitalized vinyls and magnetic tapes over a strong database, in accordance with open web standards. 
\emph{Telemeta} architecture is flexible and can easily be adapted to particular database organization of a given sound archives. 

The main features of \emph{Telemeta} are:
\begin{itemize}
\item \emph{Pure HTML} web user interface including high level \emph{search engine}
\item Smart \emph{workflow management} with contextual user lists, profiles and rights
\item RSS and JSON feed generators
\item XML serialized backup
\item Strong Structured Query Language (SQL) or Oracle backend
\item Model-View-Controller (MVC) architecture 
\end{itemize}
Beside database management, the audio support is mainly provided through an external component, TimeSide, which is described in Section~\ref{sec:Timeside}.
\defcitealias{DublinCore}{Papier I}
\subsection{Metadata}\label{sec:metadata}
In addition to the audio data, an efficient and dynamic management of the associated metadata is also required. Consulting metadata provide both an exhaustive access to valuable information about the source of the data and to the related work of peer researchers. 
Dynamically handling metadata in a collaborative manner optimises the continuous process of knowledge gathering and enrichment of the materials in the database.  
One of the major challenge is thus the standardization of audio and metadata formats with the aim of long-term preservation and usage of the different materials.
The compatibility with other systems is facilitated by the integration of the metadata standards protocols \emph{Dublin Core}\footnote{{Dublin Core} Metadata Initiative, \url{http://dublincore.org/}} and \emph{OAI-PMH} (Open Archives Initiative Protocol for Metadata Harvesting)\footnote{\url{http://www.openarchives.org/pmh/}}.

Metadata provide two different kinds of information about the audio item: contextual information and annotations.


\subsubsection{Contextual Information}
In ethnomusicology, contextual information could be geographic, cultural and musical. It could also store archive related information and include related materials in any multimedia format.

\subsubsection{Annotations and segmentation}
Metadata also consist in temporally-indexed information such as a list of \emph{time-coded markers} associated with annotations and a list of of \emph{time-segments} associated with labels. The ontology for those labels is relevant for ethnomusicology (e.g. speech versus singing voice segment, chorus, ...).

Ethnomusicological researchers and archivists can produce their own annotations and share them with colleagues. These annotations are accessible from the sound archive item web page and are indexed through the database.

It should be noted that annotations and segmentation can also be produce by some automatic signal processing analysis (see Section~\ref{sec:Timeside}).

\section{TimeSide, an audio analysis framework}\label{sec:Timeside}
One specificity of the Telemeta architecture is to rely on an external component, \emph{TimeSide}\footnote{\url{https://github.com/yomguy/TimeSide}}, that offers audio player web integration together with audio signal processing analysis capabilities. 

\emph{TimeSide} is an audio analysis and visualization framework based on both python and javascript languages to provide state-of-the-art signal processing and machine learning algorithms together with web audio capabilities for display and streaming.
Figure~\ref{fig:TimeSide_Archi} illustrates the overall architecture of \emph{TimeSide}.

\begin{figure*}[htbp]
  \centering
  \includegraphics[width=0.7\linewidth]{img/timeside_schema_v3.pdf}
  \caption{TimeSide architecture}\label{fig:TimeSide_Archi}
\end{figure*}


\subsection{Audio management}
TimeSide provides the following main features:
\begin{itemize}
\item \emph{Secure archiving, editing and publishing of audio files} over
  internet.
\item Smart \emph{audio player} with enhanced visualisation (waveform, spectrogram)
\item \emph{Multi-format support}: reads all available audio and video formats  through Gstreamer, transcoding with smart streaming and caching methods% (FLAC, OGG, MP3, WAV and WebM)
  % \item \emph{Playlist management} for all users with CSV data export
\item "On the fly" \emph{audio analyzing, transcoding and metadata
    embedding} based on an easy plugin architecture
\end{itemize}

\subsection{Audio features extraction}
In order to provide Music Information Retrieval analysis methods to be implemented over a large corpus for ethnomusicological studies, TimeSide incorporates some state-of-the-art audio feature extraction libraries such as Aubio\footnote{\url{http://aubio.org/}} \citep{brossierPhD}, Yaafe\footnote{\url{https://github.com/Yaafe/Yaafe}} \citep{yaafe_ISMIR2010} and Vamp plugins\footnote{ \url{http://www.vamp-plugins.org}}.

As a open-source framework and given its architecture and the flexibility provided by Python, the implementation of any audio and music analysis algorithm can be consider and makes it a very convenient framework.

Given the extracted features, every sound item in a given collection can be automatically analyze. The results of this analysis can be stored in a scientific file format (\emph{Numpy}, \emph{HDF5}) and serialized to the web browser through commons markup languages (\emph{xml}, \emph{json}, \emph{yaml}).


\subsection{Automatic Analysis of ethnomusicological sound archives}
Ongoing works lead by the DIADEMS project consist in implementing advanced classification, indexation, segmentation and  similarity analysis methods dedicated to ethnomusicological sound archives.

Besides music analysis, such automatic tools also deal with speech and noises classification and segmentation to enable a full annotation of the audio materials.

In the context of this project, both researchers from Ethnomusicological, Speech and Music Information Retrieval communities are working together to specified the tasks to be addressed by automatic analysis tools.


\section{Conclusion}

The Telemeta open-source framework provides a platform to provide researchers in musicology a tools to efficiently distribute, share and work on their research materials. Furthermore, this platform is offered automatic music analysis capabilities through an external component, TimeSide that provides a flexible computational analysis engine together with web serialization and visualization capabilities.

Further works on the user interface will enhance the visualization experience with time and frequency zooming capabilities and will thus improve the accuracy and the quality of time-segment base annotations.



\pagebreak

\section*{Acknowledgments} 
{\small The authors would like to thank all the people that have been involved in \emph{Telemeta} specification and development or have provide useful input and feedback. 
The project has been partially funded by the French National Centre for Scientific Research (CNRS), the French Ministry of Culture and Communication, the TGE Adonis Consortium, and the Centre of Research in Ethnomusicology (CREM).}


%\bibliographystyle{plainnat}
\bibliography{fma2014_Telemeta}


\end{document}
