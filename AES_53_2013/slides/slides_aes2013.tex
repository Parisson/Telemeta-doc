\documentclass[final, hyperref, table]{beamer}
\mode<presentation>


 %\usepackage[english]{babel} % "babel.sty"
% \usepackage{french}                  % "french.sty"
%  \usepackage{franglais}               % "franglais.sty" (a defaut)
  \usepackage{times}			% ajout times le 30 mai 2003
 
%% --------------------------------------------------------------
%% CODAGE DE POLICES ?
%% Si votre moteur Latex est francise, il est conseille
%% d'utiliser le codage de police T1 pour faciliter la césure,
%% si vous disposez de ces polices (DC/EC)
\usepackage[utf8]{inputenc}
\usepackage[T1]{fontenc}

%\usepackage{enumitem}
%\setlist{nosep}


%% ==============================================================
%\usepackage{graphicx}
\usepackage{amsmath,amsfonts}
%\usepackage[table]{xcolor}
\usepackage{subfigure}
\usepackage{fancybox}
%\usepackage{hyperref}
\usepackage{multicol}
\usepackage{wrapfig}
\usepackage{listings}
\usepackage{xcolor}

\usetheme{Boadilla}
%\usetheme{JuanLesPins}
\setbeamercovered{transparent}

%gets rid of bottom navigation bars
\setbeamertemplate{footline}[page number]{}

%gets rid of navigation symbols
\setbeamertemplate{navigation symbols}{}

%\usecolortheme{beaver}
% telemeta red
\definecolor{telemetaRed}{rgb}{0.41568, 0.01176, 0.02745}	% #6A0307
\usecolortheme[named=telemetaRed]{structure} 
\setbeamercolor{structure}{fg=telemetaRed, bg=white}
% Display a grid to help align images
%\beamertemplategridbackground[1cm]

%We will get the normal bibliography style (number or text instead of icon) by including the following code
\setbeamertemplate{bibliography item}[text]
\setbeamerfont{block title}{size=\footnotesize}
\setbeamerfont{caption}{size=\footnotesize}
% listings settings
\definecolor{lstComments}{rgb}{0,0.6,0}
\definecolor{lstBkgrd}{rgb}{1,1,0.8}
\lstset{%
  language=Python, % the language of the code
  frame=single,  % adds a frame around the code
  commentstyle=\color{lstComments},% comment style
  backgroundcolor=\color{lstBkgrd},   % choose the background color
  basicstyle=\scriptsize,       % the size of the fonts that are used for the code
  keywordstyle=\color{blue},      % keyword style
  showstringspaces=false,          % underline spaces within strings only
}
\title[TELEMETA, audio web CMS for Ethnomusicological archives]{\includegraphics[width=5cm]{logo_telemeta_alpha_black.png}\\Open web audio platform for sound archives in the use case of ethnomusicology}

\author[Fillon, Pellerin, Brossier, Simonnot]{Thomas Fillon \inst{1,2}, Guillaume Pellerin\inst{1}, Paul Brossier\inst{1}, Jos{\'e}phine Simonnot\inst{3}}


\institute[Parisson]{\tiny
  \inst{1}%
  Parisson, Paris, France\\
  \inst{2}%
  LAM, Institut Jean Le Rond d'Alembert, UPMC Univ. Paris 06, UMR CNRS 7190, Paris, France\\
 \inst{3}%
  CREM, LESC, UMR CNRS 7186, MAE, Université Paris Ouest Nanterre La Défense, Nanterre, France\\
%Thanks
\vskip1ex
 {\tiny \textcolor{red}{\emph{This work was partially done inside the DIADEMS project\\ funded by the French National Research Agency ANR (CONTINT)}}}
 \begin{center}
\hfill
   \raisebox{-4ex}{\includegraphics[width=0.1\linewidth]{../poster/img/logo_CREM.png}} \hfill
  \includegraphics[width=0.15\linewidth]{../poster/img/logo_LESC.png}\hfill
   \includegraphics[width=.3\linewidth]{../poster/img/parisson_logo_FINALE_com.pdf}\hfill
   \includegraphics[width=.18\linewidth]{../poster/img/upmc.png}\hfill
 \end{center}

}
        

\begin{document}
%\begin{frame}
%  \maketitle
%\end{frame}

\begin{frame}\tiny
\frametitle{\includegraphics[width=0.3\linewidth]{logo_telemeta_alpha_black.png} \hspace{.4\linewidth}  \includegraphics[width=.3\linewidth]{../poster/img/parisson_logo_FINALE_com.pdf}}
\framesubtitle{{\underline{DEMO SESSION}}: \textbf{Telemeta, \emph{Open web audio platform for sound archives}\\\hspace{2.7cm}\emph{in the use case of ethnomusicology}}}
% ==================================
% --------- Résumé -----------------
% ==================================
\begin{columns}

% ----------------- Colonne de gauche -----------------
\column[t]{5.5cm}

\begin{block}{The project {\tiny(started in 2007)}}
  \begin{itemize}
  \item The CREM laboratory and Parisson are developing an innovative,
    collaborative and interdisciplinary open-source web-based multimedia platform dedicated to \alert{sound archivists} and \alert{researchers in ethnomusicology}.
  \item The platform is fully operational since 2011 : \colorbox{yellow!50} { \hskip3ex \bf \url{http://archives.crem-cnrs.fr} \hskip3ex }
  \end{itemize}
\end{block}
\vspace{-0.2cm}
 \begin{block}{TimeSide :  {\tiny Open web audio processing framework}}
One specificity of the \emph{Telemeta} architecture is to rely on an external component, \emph{TimeSide}
\vspace{-0.25cm}
\begin{center}
  \colorbox{yellow!50}{\bf \hskip3ex \url{https://github.com/yomguy/TimeSide/} \hskip3ex  }
\end{center}
\vspace{-0.3cm}
%\textbf{Goals}\\
\begin{itemize}
\item Decoding, encoding (Gstreamer)
\item Asynchronous and fast audio processing (Python)
\item Feature extraction (Aubio, Yaafe, Vamp plugins)
\item Web audio player integration (HTML5), data and metadata serialization and fancy display
\end{itemize}
\end{block}


% ----------------- Colonne de droite -----------------
\column[t]{5.7cm}
\begin{block}{The platform}
  \begin{itemize}
  \item \emph{Telemeta} is a collaborative \alert{open-source audio web Content
      Management System} (CMS) dedicated to \alert{digital sound archives} and metadata publishing with index and database management
     \alert{Annotations and segmentations} (by human or automatic).
  \item \emph{Telemeta} also provides integrated \alert{audio signal
      processing tools} for automatic analysis of sound items through an external component, \emph{TimeSide}.
  \end{itemize}
\vspace{-0.5cm}
    \begin{center}
      \colorbox{yellow!50}{\textbf{\url{http://telemeta.org/}}}
    \end{center}
  \end{block}
 
  \begin{block}{Late-breaking Demos/poster session \#2 \texttt{\textbf{D2-4}}}
    \begin{itemize}
       \item Live demonstration of the \emph{Telemeta} web platform with
      the CNRS \alert{ethnomusicological sound archives}
    \item Live demonstration \emph{Timeside} audio processing framework (Python)
    \end{itemize}
  \end{block}

\end{columns}
\end{frame}

\end{document}
%%% Local Variables: 
%%% mode: latex
%%% TeX-master: t
%%% End: 
